\subsection*{\# Breif Introduction }

The \href{http://softmatter.physics.sharif.edu}{\texttt{ Soft Condensed Matter Group}} at \href{http://www.en.sharif.edu}{\texttt{ Sharif university of Technology}} lead by \href{http://sharif.edu/~ejtehadi/}{\texttt{ Prof. Mohammad Reza Ejtehadi}} has developed a unifying computational framework to create a multicomponent cell model, called the {\bfseries{Virtual Cell Model}} (V\+CM) that has the capability to predict changes in whole cell and cell nucleus characteristics (in terms of shape, direction, and even chromatin conformation) on cell substrates. Modelling data used in the package are correlated with cell culture experimental outcomes in order to confirm the applicability of the models and to demonstrate their ability to reflect the qualitative behaviour of different cells. This may provide a reliable, efficient, and fast high-\/throughput approach for the development of optimised substrates for a broad range of cellular applications including stem cell differentiation. Since the V\+CM is designed to mimic properties of soft matter in the micro scale, it can be used to study a verity of physical problems. Mechanical properties of thin film near or attached to other objects.

The V\+CM utilises 4 basic parts that are the membrane, the actin network, the nucleus, and the substrate.

\section*{The \mbox{\hyperlink{classMembrane}{Membrane}}}

The membrane is made of a series of nodes (x, y, and z coordinates of points in space) and a list of node pairs that are imported into the software. The V\+CM package can automatically import mesh files\mbox{[}$^\wedge$1\mbox{]} generated by the \href{http://gmsh.info}{\texttt{ G\+M\+SH}} software.

\mbox{[}$^\wedge$1\mbox{]}\+: The current version is compatible with the gmsh version II file style. The option is also available in gmsh versions 2 and above. 